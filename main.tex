% Version 2.20 of 2017/10/04
%
\documentclass[runningheads]{llncs}

%
\usepackage{graphicx}
\usepackage{hyperref, xcolor}
\usepackage{fancyhdr}
\usepackage{amsmath}
\usepackage{amssymb}
% Used for displaying a sample figure. If possible, figure files should
% be included in EPS format.
%
% If you use the hyperref package, please uncomment the following line
% to display URLs in blue roman font according to Springer's eBook style:
\renewcommand\UrlFont{\color{blue}\rmfamily}

\pagestyle{plain}

\begin{document}
%
\title{Estudio práctico de un algoritmo genético básico sobre el problema de la mochila}
%
\titlerunning{Problema de la mochila 0-1 \textit{Knapsack}}
% If the paper title is too long for the running head, you can set
% an abbreviated paper title here
%
\author{Javier Martínez Álvarez}
%
% First names are abbreviated in the running head.
% If there are more than two authors, 'et al.' is used.
%
\institute{Master Universitario en Investigación en Inteligencia Artificial\\
    Universidad Internacional Menendez Pelayo\\
	\email{100006269@alumnos.uimp.es}}
%
\maketitle              % typeset the header of the contribution
%
\begin{abstract}
    La importancia de los operadores de mutación y cruce en un algoritmo evolutivo (EA)
    es muy crucial para la búsqueda de la solución óptima. En este estudio se ha
    evaluado la importancia de los operadores de mutación y cruce en un algoritmo. Partiendo de 
    una implementación de un algoritmo genético de estado estacionario ya desarrrollado \cite{ref_ssga}, 
    se ha implementado la función \textit{fitness} correspondiente al problema de la Mochila 0-1.
    Usando las instancias

	\keywords{algoritmos \and metahurísticos \and genetico \and mochila \and ssga}
\end{abstract}
%
%
%
\section{Introdución}
\subsection{Problema de la mochila}
El problema de la mochila es un problema de optimización combinatoria, en el que se desea
encontrar una solución óptima para una mochila de capacidad que contiene un conjunto de
objetos, cada uno de los cuales tiene un peso.\\
Inicialmente, la mochila se encuentra vacía, y se desea maximizar el beneficio obtenido
en la selección de los objetos. La solución al problema será la suma de los beneficios de los objetos
que se incluyan en la mochila.
Por tanto, podríamos definir matemáticamente el problema de la mochila como:
\begin{equation}
    \max_{x \in {0, 1}} \sum_{j=1}^n p_j \cdot x_j
\end{equation}
\begin{equation}
    \ni {x \in {0, 1}} \sum_{j=1}^n \sum_{i=1}^m r(i,j) \cdot x_j <= b(i)
\end{equation}
Donde $n$ es es el número de objetos que se incluyen en la mochila, $m$ es el número de mochilas, $x_j$ es el valor del gen, 
$p_j$ es el beneficio que aporta el objeto, $r(i,j)$ es el peso del objeto y $b_i$ es el peso máximo de la mochila.

\subsection{Algoritmos genéticos}
El problema de la mochila se puede resolver mediante un algoritmo genético.
Este algoritmo consiste en una población de individuos, cada uno de los cuales es una solución
parcial de la mochila. Se evalúa la aptitud de cada individuo mediante la función de fitness.

\subsection{Operadores de cruce}
Los operadores de cruce consisten en unir dos individuos, mediante el cruce, para formar un
nuevo individuo. En nuestro estudio, se hará uso del operador de recombinación \textit{Single Point Crossover} \cite{ref_spx} (SPX).
Este consiste en copiar una serie de genes de un individuo padre en un individuo hijo,
y otra serie de genes de otro individuo padre en otro individuo hijo.
La partición se determina generando un número aleario y determinando si es inferior a la probabilidad de cruce.
En caso afirmativo, realizará el cruce y generará el nuevo individuo. En caso contrario, se devolverá uno de los individuos padres.

\subsection{Operadores de mutación}
Los operadores de mutación consisten en modificar un individuo modificando uno de sus genes heredados de alguno de sus padres.
A partir de una probabilidad de mutación dada por el usuario, se determina si se realizará la mutación y se intercambiará el valor de uno de los genes
por el contrario (torneo binario).

\subsection{Algoritmo genético \textit{Steady State} (ssGA)}
ssGA es una implementación de un algoritmo genético de estado estacionario. Con los siguientes pasos de forma iterativa:

\begin{enumerate}
    \item Se genera una población inicial de individuos.
    \item Se evalúa el \textit{fitness} de cada individuo de la población.
    \item Se seleccionan dos individuos haciendo uso de un \textbf{torneo binario} (\cite{ref_tournament}).
    \item Usando el operador de cruce \textbf{SPX} (\cite{ref_spx}), se genera un nuevo individuo.
    \item Al nuevo individuo, se le aplica el operador de mutación binaria.
    \item Evaluar el \textit{fitness} del nuevo individuo.
    \item Se reemplaza el nuevo individuo en la población por el individuo que obtuvo el menor valor de \textit{fitness}.
\end{enumerate}

\section{Estudio experimental}

Para realizar el sigueinte estudio experimental, se ha usado las instancias para el problema de la mochila
que se pueden encontrar en la siguiente \cite{ref_or_library}. Se ha definido un programa que nos permite leer
la instancia y ejecutar el algoritmo genético para todas los problemas de la instancia. Estos problemas fueron
definidos en el siguiente \cite{ref_balas}. 

\subsection{Ejemplo de una solución óptima}

\subsection{Estudio en la busqueda del óptimo}

\subsection{Estudio en la detención de la convergencia}

\section{Conclusiones}

% \subsubsection{Sample Heading (Third Level)} Only two levels of
% headings should be numbered. Lower level headings remain unnumbered;
% they are formatted as run-in headings.

% \paragraph{Sample Heading (Fourth Level)}
% The contribution should contain no more than four levels of
% headings. Table~\ref{tab1} gives a summary of all heading levels.

% \begin{table}
% \caption{Table captions should be placed above the
% tables.}\label{tab1}
% \begin{tabular}{|l|l|l|}
% \hline
% Heading level &  Example & Font size and style\\
% \hline
% Title (centered) &  {\Large\bfseries Lecture Notes} & 14 point, bold\\
% 1st-level heading &  {\large\bfseries 1 Introduction} & 12 point, bold\\
% 2nd-level heading & {\bfseries 2.1 Printing Area} & 10 point, bold\\
% 3rd-level heading & {\bfseries Run-in Heading in Bold.} Text follows & 10 point, bold\\
% 4th-level heading & {\itshape Lowest Level Heading.} Text follows & 10 point, italic\\
% \hline
% \end{tabular}
% \end{table}


% \noindent Displayed equations are centered and set on a separate
% line.
% \begin{equation}
% x + y = z
% \end{equation}
% Please try to avoid rasterized images for line-art diagrams and
% schemas. Whenever possible, use vector graphics instead (see
% Fig.~\ref{fig1}).

% \begin{figure}
% % \includegraphics[width=\textwidth]{fig1.eps}
% \caption{A figure caption is always placed below the illustration.
% Please note that short captions are centered, while long ones are
% justified by the macro package automatically.} \label{fig1}
% \end{figure}

% \begin{theorem}
% This is a sample theorem. The run-in heading is set in bold, while
% the following text appears in italics. Definitions, lemmas,
% propositions, and corollaries are styled the same way.
% \end{theorem}
% %
% % the environments 'definition', 'lemma', 'proposition', 'corollary',
% % 'remark', and 'example' are defined in the LLNCS documentclass as well.
% %
% \begin{proof}
% Proofs, examples, and remarks have the initial word in italics,
% while the following text appears in normal font.
% \end{proof}
% For citations of references, we prefer the use of square brackets
% and consecutive numbers. Citations using labels or the author/year
% convention are also acceptable. The following bibliography provides
% a sample reference list with entries for journal
% articles~\cite{ref_article1}, an LNCS chapter~\cite{ref_lncs1}, a
% book~\cite{ref_book1}, proceedings without editors~\cite{ref_proc1},
% and a homepage~\cite{ref_url1}. Multiple citations are grouped
% \cite{ref_article1,ref_lncs1,ref_book1},
% \cite{ref_article1,ref_book1,ref_proc1,ref_url1}.


% ---- Bibliography ----
%
% BibTeX users should specify bibliography style 'splncs04'.
% References will then be sorted and formatted in the correct style.
%
\bibliographystyle{splncs04}
\bibliography{mybibliography}
%
\begin{thebibliography}{8}
    \bibitem{ref_ssga}
	Author, E., Author: Steady State Genetic Algorithm (ssGA),
	(1999). \url{https://neo.lcc.uma.es/software/ssga/index.php}

    \bibitem{ref_spx}
	Author, T., Author, S., Author, M.: Theoretical Analysis of Simplex Crossover for Real-Coded Genetic Algorithms,
	France (2000). \url{http://www.springer.com/engineering/computational-systems-control/theoretical-analysis-of-simplex-crossover-for-real-coded-genetic-algorithms}

    \bibitem{ref_tournament}
	Tournament selection,
	(2021). \url{https://en.wikipedia.org/wiki/Tournament_selection}

    \bibitem{ref_balas}
	Author, C. C. :Computational experience with variants of the Balas algorithm applied to the selection of R\&D projects,
	(1967). \url{https://en.wikipedia.org/wiki/Tournament_selection}

    \bibitem{ref_or_library}
	Author, J., Author: OR Library,
	\url{http://people.brunel.ac.uk/~mastjjb/jeb/orlib/mknapinfo.html}


\end{thebibliography}

\end{document}
